\documentclass[12pt]{article}
\usepackage[spanish]{babel}
\usepackage{amsmath}
\usepackage{graphicx}
\usepackage{url}
\begin{document}

\begin{center}
\bf{\sc\Huge Universidad de Antioquia}\\
\end{center}
\vspace{120pt}
\begin{center}
\bf{\sc\Huge Diana Sofia Salgado Andrade }\\
\end{center}
\vspace{200pt}
\begin{center}
\bf{\sc\Huge Medellín}
\end{center}
\begin{center}
\bf{\sc\Huge 2020}\\
\end{center}\
\newpage



\begin{center}

\section{ Introducción}
\end{center}
\begin{flushleft}
\vspace{25PT}
\large
La historia de la computación esta marcada por grandes avances a través del tiempo, la creación de tarjetas perforadas, tubos de vacío y el gran salto que llevo a todo lo que conocemos actualmente, los transistores. Pero  nada de esto habría sido posible gracias a grandes personajes que desde décadas pasadas, buscaban la solución a los grandes problemas matemáticos, uno de ellos el infinito, tema que desencadeno toda una cadena de preguntas y dudas que buscaban respuesta, teoremas que parecían estar cerca de la solución pero a la final, eran derribados por nuevos ideales, afirmaciones y demostraciones, pero que de no ser por esto seguramente no podríamos gozar de la maravilla de la ciencia de la computación. ¿Qué paso exactamente desde que la idea del infinito intrigara a tantos intelectuales?  ¿Como logró esta idea, ser el origen de esta revolución matemática?
\end{flushleft}
\newpage


\begin{center}
\section{ DE LA CRISIS A TU ORDENADOR}
\end{center}
\begin{flushleft}
\vspace{25PT}
\large
La idea del infinito viene rondando la mente de los matemáticos desde la antigua Grecia, Aristóteles (1995, 384-322 A.C.) lo definía en muchos ámbitos tanto físicos como mentales, siendo según él la razón más poderosa el que el ser humano no encuentra limite en su pensamiento. \cite{Aris}
Años después, en 1874, el matemático George Cantor quien desde su juventud se interesó en el infinito, abarcó este tema desarrollando y publicando los artículos que conformarían su Teoría de Conjuntos, en la que Cantor determina un conjunto como la colección de objetos que pueden ser finitos o infinitos, precisó el concepto de “cardinal” que refiere al numero de objetos que conforman ese conjunto, ademas, definió que no todos los conjuntos infinitos tienen el mismo cardinal, es decir, conjuntos infinitos de diferente tamaño. Con estas y mas revelaciones que complementan su Teoría de Conjuntos, acabó desarrollando la aritmética transfinita completa, esta ocupaba la operación de suma y multiplicación del conjunto de los números naturales a los cardinales infinitos que trata en dicho teorema.
\cite{Cantor}


\vspace{10PT}


\vspace{15PT}
Todo esto llevo a la llamada Crisis de los Fundamentos, un punto de quiebre para muchos en el ámbito matemático, dividió a los intelectuales en diferentes posiciones frente a los diversos temas y problemáticas que hacían tambalear todo lo que se conocía como las bases de la ciencia matemática, de este movimiento cabe resaltar a David Hilbert, un muy conocido matemático de la época, Hilbert esperaba poder crear un conjunto de axiomas que sustentaran todas las paradojas matemáticas, estaba convencido de que las matemáticas tenían que ser totalmente infalibles ,que no podían fallar, cosa que no pudo demostrar en su totalidad y que a la final, se vio acabado cuando en un congreso matemático, los espectadores se vieron sorprendidos por la participación de un joven matemático, Kurt Gödel, este revelo que la discusión y lo que se proponía en el programa de Hilbert estaba por terminar, que no había ningún grupo de axiomas o un algoritmo que pudiera comprobar o sustentar todas las problemáticas propuestas. Tiempo después, Gödel anuncio sus Teoremas de la Incompletitud en los que demuestra que ciertas teorías aritméticas simplemente son imposibles de demostrar mediante otros axiomas o medios.

\vspace{15PT}
El legado de Gödel fue seguido por Alan Turing, conocido principalmente por su aporte a la creación de “Enigma”, clave de la victoria en la Segunda Guerra Mundial, ya que descifró los mensajes encriptados enviados por los alemanes para coordinar sus ataques, pero realmente Turing logro ser reconocido como el padre del ordenador moderno, a pesar de su corta vida, sus aportes a la ciencia de la computación no tienen igual. Gracias a la creación de la maquina universal, basada en la clásica maquina de Turing que era capaz de implementar cualquier algoritmo siguiendo una cinta con un orden de reglas marcadas con símbolos, esta, al igual que la maquina antes nombrada, se trataba de la simple cinta y un escáner, esta cinta contenía también símbolos ya fueran ceros, unos o cualquier cosa, era leída por el escáner y se ejecutaban las reglas que estos símbolos representaban, esta idea dio pie para el principio básico de todos los computadores, el algoritmo básico del que se partiría para nombrarlo el concepto generador de todos los ordenadores.
 \end{flushleft}
 \newpage

 \bibliography{other/biblio}


 








\end{document}
